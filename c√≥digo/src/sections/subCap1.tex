
\chapter{SubCap1 Prueba de contenidos}

\section{Algoritmicidad ideas preliminares}

¿Qué significa lo experimental en el contexto de los algoritmos? ¿Cuáles son los límites entre humano y máquina, cuáles son las consideraciones en términos de autoría e intención? ¿qué tipo de relación existe entre algoritmos y cuerpos, podríamos hablar de algoritmicidad corporea o cuerpo algorítmico? Este texto explora a través de tres obras (sonoras, performáticas) los espacios experimentales y de agencialidad presentes en los algoritmos. Dichas perspectivas buscan prevenir una estabilización y oscilaciones dirigidas a la simple demostración del algoritmo a través de una prueba autocontenida que lo llevaría a perder su capacidad investigativa. Se parte de la idea de que los algoritmos (incluído también el hardware) no son entidades estructuradas fijas en el tiempo-espacio-materia sino que aparecen impermanentes a través de la posibilidad intrínseca a su ejecución, manipulación transformación y actualización ad infinitum. En vez de apostar por una estética probabilística y predictiva se partirá de una “tendencia especulativa intrínseca de la computación, que produce una novedad genuina incapaz de explicarse por fuerzas externas o condiciones iniciales” (Parisi, 2013). Se entiende al algoritmo como un objeto técnico integrado por procesos lógicos y de control que articula un medio/espacio de algoritmicidad o agencialidad algorítmica. Esta dicotomía incluye los principios que animan los algoritmos es decir la capacidad de interactuar y experimentar con ellos y entenderlos como vehículos de pensamiento además de su propia relación con los procesos de intercambio/afectación (“intra-acción”) con su entorno (sea digital o físico). A su vez, la “agencialidad crítica”, es decir una aproximación que analiza permanentemente cómo, cuándo, dónde y porqué se produce la agencia, es una interfaz entre el investigador y su aparato, una herramienta epistémica que serviría para desdibujar cada vez más las fronteras y reforzar la injerencia/proximidad/entrelaza-miento de la máquina en el artista-investigador y viceversa (extimacy Rheinberger 2013, intra-action Barad 2012). Así pues, se estudiarán las relaciones entre manos y mentes, técnicas e ideas, materias y formas reforzando el componente experimental para conceptualizar tanto errores como aciertos que promuevan la construcción de un pensamiento artístico-tecno-científico comprometido con la generación de nuevos conceptos, teorías e hipótesis.\footnote{Kowalski (1979): “Algoritmo = Lógica + Control”. La parte lógica se compone de definiciones de procedimientos abstractos relacionados con el conocimiento sobre el dominio del problema y de las estructuras de datos sobre las que operan estos procedimientos, mientras que la parte de control se ocupa de estrategias para convertir el componente lógico en una máquina eficiente, estrategias para desenrollar el conocimiento en el tiempo y el espacio.}

\section{Código y algoritmo}
Esta sección del texto vincula las ideas presentadas en la sección anterior con el estudio del código. A partir de ver un algortimo desde la pesperctiva de su código surgen las siguientes preguntas: ¿qué relación tiene una algoritmo con su código? y ¿cómo se expresan a través de una práctica musical computacional? Para abordar estas preguntas, que son puntos iniciales para entender este vínculo, partimos de la idea de que el código computacional es una serie de símbolos y caracteres legibles por el humano; de forma específica nos refereimos al código fuente. Sin embargo, en este artículo no es la intención analizar el código fuente de un algortimo sino aquello para lo que está diseñado.

El interés de este texto recae en las prácticas y técnicas musicales que se desprenden del cruce con las ciencias de la computación, como el live coding y el aprendizaje de máquina. Si bien hay un campo de la tecnología musical que estudia el aprendizaje de máquina en el live coding (Xambó, Bernardo et al.), nuestro interés viene de la práctica artística y el desarrollo tecnológico que realizamos en nuestras respectivas investigaciones. En este sentido, la aproximación a la creación musical en la que intervienen humanos y máquinas es desde dos perspectivas es a partir es de el proyecto de investigación de cada uno. Desde mi perspectiva el estudio del live coding centrado en la creatividad humana con lenguajes de computación o programación creativa que va a dialogar con una perspectiva que estudio, entre otras cosas, centrada en la co-creación humano-máquina de la improvisación libre y el aprendizaje de máquina.

En este contexto, podemos pensar, por un lado, en un panorama que ve al humano al centro de la creatividad con computadoras, y por otro a la máquina como agente creativo o colaborador del humano en la creación musical. Este escenario nos deja ver diferentes aproximaciones a la relación humano-maquina en la creación musical, desde la colaboración entre humanos y agentes de inteligencia artificial hasta la creación autónoma de las máquinas.


Referencias

Kleiman, A (2012). Interactions: an interview with Karen Barad, Mousse Magazine, (34) 76-81.
Kowalski, R. (1979). Algorithm = logic + control. Communications of the ACM, 22(7), 424–436.
Rheinberger, H-J. (1998) Experimental systems, graphematic spaces. In T Lenoir (Ed.), Inscribing Science: scientific texts and the materiality of communication (pp 285-303). Palo Alto: Stanford University Press.
Manovich, L. (2013). Software takes command. New York: Bloomsbury Academic.
Parisi, L. (2013). Contagious architecture: computation, aesthetics, and space. Cambridge, MA:
MIT Press.
Rheinberger, H.-J. (2013). Forming and Being Informed: Hans-Jörg Rheinberger in conversation with Michael Schwab. In M. Schwab (Ed.), Experimental systems. future knowledge in artistic research (pp. 198–219). Leuven: Leuven University Press.
Schwab, M. (Ed.). (2013). Experimental systems. future knowledge in artistic research. Leuven: Leuven: Lueven University Press.

