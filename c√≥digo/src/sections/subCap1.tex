\chapter{La capacidad investigativa del algoritmo y el código: trazos sobre el ciclo práctica, tecnología e investigación}

%autores de capítulo
\author{Aaron Castañeda y Hernani Villaseñor}

\section{Introducción}
Escribir un texto colaborativo implica negociar perspectivas, ideas, términos y conceptos, antes se requiere llegar a un acuerdo sobre el tema a desarrollar. El inicio de este texto nos coloca frente a un basto panorama de ideas que surgen de nuestras investigaciones en curso, las cuales tratan sobre el aprendizaje de máquina y el live coding. Como estrategia hemos escrito párrafos intercalados de los cuales sustrajimos términos que nos parecen relevantes. Aunque el resultado de este procedimiento es extenso y poco acotado nos impulsó a escribir y detectar algunas temáticas que conectan las dos investigaciones.

Como primer punto hemos encontrado que nos preocupa lo que llamamos ‘capacidad investigativa’ del algortimo y el código; y como segundo punto el ciclo de retroalimentación por el que pasa el artista-investigador en su investigación. Dicho ciclo lo vemos de diferente forma: uno se comopone por la práctica artística, el desarrollo tecnológico y la investigación, el otro por un pensamiento artístico, técnico y científico. A partir de estas aproximaciones trataremos de pecisar, organizar y desarrollar nuestro texto con ideas sobre agencia, autoría, medio o interfaz, creatividad computacional, programación creativa, materialidad y abstracción.

\section{La condición del artista investigador}
Comenzamos por definir nuestra condición de artistas e investigadores para aclarar el contexto en el que escribimos el presente capítulo. En el momento de escribir este texto ambos autores nos encontramos realizando el doctorado en música en el área de tecnología musical. Nuestras investigaciones, aunque tratan temas distintos dentro del ámbito de la música computacional, están conectadas por la aproximación a la investigación desde la práctica artística, el desarrollo de tecnología y la problematización de ambos aspectos. En este sentido, escribimos sobre casos que se transforman conforme avanza cada investigación cuya materia está constituida por algoritmos escritos con código computacional. Por un lado, estos algoritmos están aplicados al aprendizaje de máquina; por otro lado, a la escritura y modificación en vivo.

\section{Algoritmicidad ideas preliminares}

¿Qué significa lo experimental en el contexto de los algoritmos? ¿Cuáles son los límites entre humano y máquina, cuáles son las consideraciones en términos de autoría e intención? ¿qué tipo de relación existe entre algoritmos y cuerpos, podríamos hablar de algoritmicidad corporea o cuerpo algorítmico? Este texto explora a través de tres obras (sonoras, performáticas) los espacios experimentales y de agencialidad presentes en los algoritmos. Dichas perspectivas buscan prevenir una estabilización y oscilaciones dirigidas a la simple demostración del algoritmo a través de una prueba autocontenida que lo llevaría a perder su capacidad investigativa. Se parte de la idea de que los algoritmos (incluído también el hardware) no son entidades estructuradas fijas en el tiempo-espacio-materia sino que aparecen impermanentes a través de la posibilidad intrínseca a su ejecución, manipulación transformación y actualización ad infinitum. En vez de apostar por una estética probabilística y predictiva se partirá de una “tendencia especulativa intrínseca de la computación, que produce una novedad genuina incapaz de explicarse por fuerzas externas o condiciones iniciales” (Parisi, 2013). Se entiende al algoritmo como un objeto técnico integrado por procesos lógicos y de control que articula un medio/espacio de algoritmicidad o agencialidad algorítmica. Esta dicotomía incluye los principios que animan los algoritmos es decir la capacidad de interactuar y experimentar con ellos y entenderlos como vehículos de pensamiento además de su propia relación con los procesos de intercambio/afectación (“intra-acción”) con su entorno (sea digital o físico). A su vez, la “agencialidad crítica”, es decir una aproximación que analiza permanentemente cómo, cuándo, dónde y porqué se produce la agencia, es una interfaz entre el investigador y su aparato, una herramienta epistémica que serviría para desdibujar cada vez más las fronteras y reforzar la injerencia/proximidad/entrelaza-miento de la máquina en el artista-investigador y viceversa (extimacy Rheinberger 2013, intra-action Barad 2012). Así pues, se estudiarán las relaciones entre manos y mentes, técnicas e ideas, materias y formas reforzando el componente experimental para conceptualizar tanto errores como aciertos que promuevan la construcción de un pensamiento artístico-tecno-científico comprometido con la generación de nuevos conceptos, teorías e hipótesis.\footnote{Kowalski (1979): “Algoritmo = Lógica + Control”. La parte lógica se compone de definiciones de procedimientos abstractos relacionados con el conocimiento sobre el dominio del problema y de las estructuras de datos sobre las que operan estos procedimientos, mientras que la parte de control se ocupa de estrategias para convertir el componente lógico en una máquina eficiente, estrategias para desenrollar el conocimiento en el tiempo y el espacio.}

\section{Código y algoritmo}
Esta sección del texto vincula el término código fuente con las ideas acerca del término algoritmo presentadas en la sección anterior. A partir de ver un algortimo desde su código surgen las siguientes preguntas: ¿qué relación tiene una algoritmo con su código? y ¿cómo se expresan a través de una práctica musical computacional? Para abordar estas preguntas partimos de la definición de que el código computacional es una serie de símbolos y caracteres legibles por el humano y posteriormente, al ser traducidos, ejecutables por la máquina. En este artículo no es la intención analizar el código fuente de un algortimo sino aquello hace en el contexto de una práctica musical. En este sentido, el código fuente es una forma de aproximarnos al análisis del algortimo.

El interés de este texto recae en las prácticas y técnicas musicales que se desprenden del cruce entre las ciencias de la computación y la música, como el live coding y el aprendizaje de máquina. Si bien hay un campo de la tecnología musical que estudia el aprendizaje de máquina aplicado al live coding (Xambó, Bernardo et al.), nuestro interés viene de la práctica artística y el desarrollo tecnológico que realizamos en nuestras respectivas investigaciones. Por un lado, el desarrollo de un sistema de improvisación a partir de las escucha de máquina y por el otro, el desarrollo de una práctica artística de live coding basada en el desarollo de una librería de captura de sonido en el momento. Esta última perspectiva estudia el live coding centrado en la creatividad humana con lenguajes de programación sonora que va a dialogar con una perspectiva de estudio centrada en la co-creación humano-máquina en la improvisación libre y el aprendizaje de máquina.

En este contexto, podemos pensar, por un lado, en un panorama que ve al humano al centro de la creatividad con computadoras, y por otro a la máquina como agente creativo o colaborador del humano en la creación musical. Este escenario nos deja ver diferentes aproximaciones de la relación humano-máquina en la creación musical con lenguajes de programación, desde la colaboración entre humanos y agentes de inteligencia artificial hasta la creación autónoma de las máquinas.


\section{Referencias}

Baalman, M. (2015). Embodiment of code.

Kleiman, A (2012). Interactions: an interview with Karen Barad, Mousse Magazine, (34) 76-81.

Kowalski, R. (1979). Algorithm = logic + control. Communications of the ACM, 22(7), 424–436.
Rheinberger, H-J. (1998) Experimental systems, graphematic spaces. In T Lenoir (Ed.), Inscribing Science: scientific texts and the materiality of communication (pp 285-303). Palo Alto: Stanford University Press.
Manovich, L. (2013). Software takes command. New York: Bloomsbury Academic.
Parisi, L. (2013). Contagious architecture: computation, aesthetics, and space. Cambridge, MA:
MIT Press.
Rheinberger, H.-J. (2013). Forming and Being Informed: Hans-Jörg Rheinberger in conversation with Michael Schwab. In M. Schwab (Ed.), Experimental systems. future knowledge in artistic research (pp. 198–219). Leuven: Leuven University Press.
Schwab, M. (Ed.). (2013). Experimental systems. future knowledge in artistic research. Leuven: Leuven: Lueven University Press.
Xambó, A., Lerch, A. y Freeman, J. (2019). Music Information Retrieval in Live Coding: A Theoretical Framework. Computer Music Journal, 42(4), 9-25.

